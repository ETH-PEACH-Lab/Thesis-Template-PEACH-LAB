\section{Related Work}
The primary function of Related Work is to answer the following four questions: 
\begin{itemize}
    \item Who else has done work with relevance to this work of yours? \footnote{This can be a research paper, an open-source project, or a commercial product.}
    \item What did they do?
    \item What did they find?
    \item And how is your work here different?
\end{itemize}

This last step is called \emph{differentiation}.

(Sidebar: You will find that well-written Abstracts in others’ papers answer \say{What was done?} and \say{What was found?} and are immensely valuable to you as you scour the literature. 
In contrast, Abstracts that are mostly motivation or describe their paper, not their work, require you to look into the paper itself to have any idea what was actually done.
Such Abstracts are very unhelpful, but sadly, exceedingly common.)

Related Work should not read like a laundry list of who--did--what.
Related Work should offer insights and education
about prior work.
It should \emph{teach} readers something and help them understand the prior work better than they did before.

Skillfully written Related Work sections will often group prior work into themes. 
In 10-page papers, these themes may be subsections. 
Differentiation of the current work from prior
work can then be achieved on a per-theme basis, rather than differentiating against every piece of prior work raised.

Differentiation should not be defensive. It is not required that the current work assert itself as \say{better} than prior work. 
That is for reviewers to decide. Rather, current work must be \emph{different} than prior work. 
It must ask a different question, use a different method or technique, be built on different technology, or have different results. 
(The one exception to differentiation in this way is a pure replication study, although even then, there are usually attempts to augment or supplement the methods used.)