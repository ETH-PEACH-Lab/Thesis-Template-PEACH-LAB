\section{Design of Your Widget}
After Related Work, the next sections begin the description of your work: \emph{What was done?} 
If you are using design-driven research, you will often begin by describing what you created. 
If you have a \sysname{}, this section often takes 2--3 full pages to describe it in a 10-page paper. 
This section should include things like the goals for the \sysname{}, the principles underlying it, the optimizations you favored, the tradeoffs you faced, the rationale for the choices you made, the design or engineering process you went through, how and why the Created Thing works, how it is built, its most important properties, its appearance, its function, how and in what situations is it used, and what its limitations are. 
Basically, you are trying to give a full, replicable--by--an--expert account of what you created, your \sysname{}.

Figures and diagrams should be used liberally throughout this section to illustrate key points. 
I personally prefer figures to be inline and between paragraphs, inserted just after the paragraph that first refers to them. 
(The only exception to this is Figure \ref{fig:enter-label}.) 
That way, the reader encounters the figures \say{in the flow of reading,} rather than out of sequence and having to jump their eyes back and forth from the text to the figure.

\subsection{Subsections}
Subsections are almost always used in these sections. 
Near the end of the section it is often reasonable to have an Implementation Details subsection describing things like how many lines of code are in your \sysname{} (if the thing involves software), what programming language it was written in, what platform it runs on, and related. 
Such subsections are not particularly interesting but for technical readers, they can help to illuminate aspects of the innovation.
But the main focus of these write-ups should not be on fungible implementation details; focus instead on the ideas behind the innovation and the ways those ideas were
realized.

\subsubsection{Subsubsections}
Unless you absolutely must use them, avoid using subsubsections. 
Even the word \say{subsubsections} is
unpleasant to read, let alone what is in them. \footnote{April: I slightly disagree with this. If a subsubsection makes your paper clearer to read, go ahead and use it.}