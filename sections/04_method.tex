\section{Method (But don't just name it ``Method'')}

Regardless of whether you are reporting on design-driven research or scientific research (meaning, whether or not you have the previous section describing a Created Thing or not), you will often have a method section describing the research method(s) by which you carried out your scientific work.

(There are exceptions to this structure, such as engineering--oriented papers that provide no empirical validation of the \sysname{} save the fact that they made the \sysname{} exist in the first place. 
So-called \say{existence proof} papers are
entirely legitimate; in such cases, most of those papers will be filled with descriptions of the \sysname{}, and any study of that thing will be analytical, not empirical. 
The Method section may therefore become instead an analytical assessment of the \sysname{}’s properties, rather than a description of any empirical methods.)

For papers containing empirical scientific accounts of some phenomenon, the research methods employed must be described carefully and in great detail. 
These research methods may be quantitative, qualitative, or mixed, and they should follow established norms in related fields such as psychology, sociology, or anthropology. 
The following subsections give method outlines for two common study types, quantitative laboratory experiments and qualitative field studies.

\subsection{Method Outline for Quantitative Laboratory Experiments}
If you are writing up your method for a quantitative laboratory experiment, a common structure found in many psychology articles is as follows:

\emph{Participants:} Describe how many participants took part, their genders, their mean age (M=32.3 years), their age variance (SD=6.2 years), and any other important characteristics. 
If any inclusion or exclusion criteria were used, report them.
The means by which participants were sampled and recruited should also be reported. 
If participants were compensated, say how. 
Give enough detail that a similar group of participants could be recruited in the future by an expert reader.

\emph{Apparatus:} Describe the experiment setup, including what equipment was used, how it was arranged, how the laboratory was set up, and so on. 
Also describe whatever software was running and the computer system it was running on, including the operating system and version, the hardware make and model, and anything else relevant to the particular technology used, e.g., frames per second for video recording.

\emph{Procedure:} Often the longest subsection within Method, the Procedure subsection should describe the process participants went through from their arrival to their departure from the lab. What tasks did they perform? In what order?
How many? What constituted a single “trial?” Enough detail should be provided that an expert reader could re-run the experiment if they were given the same participants and apparatus.

\emph{Design \& Analysis:} This subsection describes the experiment formally, using one paragraph to describe the experiment design in statistical terms, and one paragraph to describe the statistical analysis. 
An effective technique for experiments with multiple factors is to lay out the factors and their levels in a list. 
For example, we might write, \say{The experiment was a 3×2×2 within-subjects design with the following factors
and levels}:
\begin{quote}
\begin{itemize}
    \item \emph{Devices}: mouse, trackball, touchpad;
    \item \emph{Desk}: sitting desk, standing desk;
    \item \emph{Illumination}: lighted, darkness;
\end{itemize}    
\end{quote}

The analysis paragraph covers the formal statistical analysis approach used. 
For example, it might report: \say{We collected
148 data points for each of 16 participants, giving us 2368 data points in all. 
Twenty-two of these data points were removed due to sensor failures, resulting in 2346 viable data points for analysis.
The analysis was carried out with a repeated measures ANOVA using Bonferroni adjustments for post hoc comparisons and the Greenhouse-Geisser correction for violations of sphericity.}
As with other subsections, the test here is whether an expert reader could repeat the analysis with the description given.

\subsection{Method Outline for Qualitative Field Studies}
Another common type of study is the qualitative field study.
Typically, such studies are reported using a wider range of structures than for laboratory experiments. 
Nevertheless, there are certain essential methodological aspects to cover when reporting a qualitative field study. 
You can also seek best practices from sociological or anthropological societies and journals.

\emph{Participants}: Your method report should include coverage of the participants in the study. 
Depending on your study type and terminology preference, these may be informants, respondents, or just generally participants. A thick description of these participants is important, much more so than for the laboratory experiment. 
How did you gain access to the participants?
Where and who were they? 
What roles did they occupy in the setting you were in? 
These are just a few of the questions to address about your participants.

\emph{Theory}: Research built on theory may describe how the field study was designed in terms of the theoretical constructs underpinning it. 
Often, theory serves as an aid to decisionmaking when designing any kind of study, and that rationale can be invoked in the study description: 
Why were certain participants chosen and not others? 
Why was a particular field setting chosen? 
Why was a certain research method preferred? 
What does theory say about the phenomena being
studied and how it should be studied?

\emph{Procedures}: The procedures used in the field should also be described. 
These procedures may be about how the data were
gathered or recorded, how informants or settings were chosen and accessed, what questions were asked, what artifacts were gathered, what probes were used, how much time was spent with certain participants or at various sites, and where and how certain observations were made. 
Was an ethnographic account the goal? 
Was participant observation or non-participant observation used? How? 
The one type of procedure that is usually covered separately is the procedure for data analysis, which is next.

\emph{Analysis}: How were the qualitative data analyzed? 
How was objectivity and validity ensured? Many formal qualitative studies use grounded theory as an analysis method. 
If so, then the procedures that you followed for open coding, axial coding, selective coding, and theory formation should be
described in detail. 
Give a representative sample of your codes.
You could also give a link to an online version of a full coding manual. 
Was inter-rater reliability assessed?
How? 
And what was the outcome? 
Give enough detail that the analysis could be replicated by an expert reader if he or
she had your data.
