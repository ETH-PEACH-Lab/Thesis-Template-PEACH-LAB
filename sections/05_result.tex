\section{Results}
At last we reach the exciting part, your Results! 
Your Results section speaks for itself: report on the results of your work in an organized way. 
Refrain from reporting on the significance of these results until the Discussion section, which comes next. 
For now, report on the results dispassionately. 
Use charts, graphs, and tables as appropriate. 
Good Results sections do not simply jam a bunch of numbers in, but \say{tell a story with data,} creating an easy-to-read narrative flow that does not make the reader do too much work to figure out what happened.

For quantitative laboratory studies, often the Results section is divided into subsections by dependent variable, i.e., the things that were measured. 
When including the results of statistical tests, do not just say that there was a significant effect, interpret that effect for the reader. 
\begin{quote}
    Worse: \say{There was a significant main effect of Illumination on pointing time (F(1,28)=15.79, p<.0001).} 
\end{quote}
\begin{quote}
    Better: \say{There was a significant main effect of Illumination on pointing time (F(1,28)=15.79, p<.0001), as lighted conditions caused faster pointing than darkened conditions.}
\end{quote}

For qualitative field studies, often the Results section is divided into subsections around emergent themes. 
Results sections may be quite long, incorporating numerous observations and direct quotes from participants. 
Organizing your results in subsections is key to creating an easy--tofollow story.