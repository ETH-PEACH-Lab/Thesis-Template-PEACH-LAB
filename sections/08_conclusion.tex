\section{Conclusion (Not Conclusion{\color{red}s})}
Your Conclusion is the section in which you: 
(1) affirm that you have delivered on the claims made in your Introduction,
(2) summarize the contributions of the work, 
(3) make any key points with which you would like to leave the reader, and
(4) point to a bright future, a better world, for your work having been done in it.
It should not be necessary to re-report the key findings of the work, although doing so to a limited extent can be okay. 
The findings have already been reported, so it is better to \say{zoom out} and report on the key contributions of the work.
\begin{quote}
    Worse: \say{We showed in this work that illuminated office conditions result in an 84\% improvement in pointing speed than dark offices.}
\end{quote}
\begin{quote}
    Better: \say{This work contributed the first empirical study of pointing under different office illumination conditions.}
\end{quote}
\begin{quote}
    Unnecessary: \say{This work contributed the first empirical study of pointing under different office illumination conditions, finding that illumination improves pointing speed by 84\% over dark offices.}
\end{quote}
    
Try to frame the contributions of the work such that they speak to your broader scholarly community, not just those interested in your narrow topic. 
Imagine someone from the popular press reading your Conclusion. 
Could they imagine a news story on your work from what they read?

Now that you have been given numerous tips on how to structure your research paper, I am confident that you will do right by the good research you are doing by writing it in ways that others can understand and appreciate. 
The world will be made better for all the work you do and the way you communicate it. 
Go forth and write well!
